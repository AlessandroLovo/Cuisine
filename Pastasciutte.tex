% LTeX: language=it_IT

\documentclass[main.tex]{subfiles}
\graphicspath{{Figures/Pastasciutte/}}

\begin{document}
    \chapter{Pastasciutte}
    
    \newpage
    \begin{recipe}
        [%
            preparationtime = {\unit[45]{min}},
            portion = {\unit[1]{Porzione}},
            source = {Lyon 20/10/2021},
        ]
        {Spaghetti alla zucca e zucchine}

        \graph{
            big = SpaghettiZuccaZucchine.JPG
        }

        \ingredients{
            \unit[120]{g} & Spaghetti \\
            1/2 & Cipolla \\
            \unit[150]{g} & Zucca \\
            1 & Zucchina \\
            \unit[1]{cucchiaio} & Olio EVO \\
            \unit[1]{bicchiere} & Acqua \\
            \unit[1]{pizzico} & Sale \\
            \unit[1]{pizzico} & Zucchero \\
            \unit[1]{spolverata} & Peperoncino \\
            \unit[1]{pezzo} & Scorza di limone \\
        }

        \preparation{
            \begin{enumerate}
            \item Mondate la cipolla e tagliatela a pezzettini piccoli. Scaldate l'olio in una padella antiaderente e soffriggete.
            \item Tagliate a cubetti piccoli la zucca e aggiungetela alla padella. Fate saltare per un minuto circa, poi aggiungete gradualmente l'acqua. Il quantitativo di acqua dipende da quanto è alto il fuoco e se cucinate con o senza coperchio. L'importante è non annegare il contenuto ma neanche bruciarlo. Nel frattempo tagliate la zucchina a julienne e scaldate l'acqua per la pasta.
            \item Quando la zucca comincia ad ammorbidirsi aggiungete la zucchina, sale, zucchero e peperoncino.
            \item Finché vi dedicate alla cottura degli spaghetti controllate che le zucchine non si spappolino. Se cominciano a diventare troppo molli, semplicemente spegnete il fuoco.
            \item Finché la pasta cuoce sminuzzate la scorza di limone e aggiungetela al sugo.
            \item Quando gli spaghetti sono pronti scolateli (meglio un po' al dente) e poi saltateli velocemente con il sugo preparato.
            \end{enumerate}
        }
    \end{recipe}

    \newpage
    \begin{recipe}
        [%
            preparationtime = {\unit[30]{min}},
            portion = {\unit[1]{Porzione}},
            source = {Lyon 22/10/2021},
        ]
        {Spaghetti cacio e kiwi}

        \graph{
            big = SpaghettiCacioKiwi.JPG
        }

        \ingredients{
            \unit[120]{g} & Spaghetti \\
            1/4 & Cipolla \\
            2 & Kiwi \\
            \unit[40]{g} & Pecorino sardo stagionato \\
            \unit[1]{cucchiaio} & Olio EVO \\
            \unit[1]{pizzico} & Zucchero \\
            \unit[1]{spolverata} & Peperoncino \\
            \unit[1]{rametto} & Timo \\
        }

        \preparation{
            \begin{enumerate}
            \item Mondate la cipolla e tagliatela a pezzettini. Scaldate l'olio in un pentolino antiaderente e soffriggete.
            \item Sbucciate e tagliate a pezzettini i due kiwi aggiungeteli alla padella. Fate saltare per un minuto circa, poi aggiungete un goccio d'acqua, lo zucchero il timo e il peperoncino. Mescolate e coprite con un coperchio. Cuocete a fuoco medio basso finché non ottenete una pseudo-poltiglia.
            \item Nel frattempo cuocete la pasta e grattugiate il pecorino.
            \item Quando la pasta è pronta scolatela e impiattate mescolandovi il sugo e il pecorino grattuggiato.
            \end{enumerate}
        }
    \end{recipe}

    

    \newpage
    \begin{recipe}
        [%
            preparationtime = {\unit[1]{h}},
            portion = {\unit[1]{Porzione}},
            source = {Lyon 13/01/2022},
        ]
        {Fusilloni con zucchine alla curcuma}

        \graph{
            big = FusilloniZucchineCurcuma.jpeg
        }

        \ingredients{
            \unit[120]{g} & Fusilloni \\
            1/2 & Cipolla \\
            1/2 & Zucchina \\
            \unit[1]{spicchio} Aglio \\
            10 & Pomodorini \\
            \unit[2]{cucchiai} & Olio EVO \\
            \unit[1]{pizzico} & Sale \\
            \unit[1]{pizzico} & Peperoncino \\
            \unit[1]{cucchiaino} & Curcuma \\
            & Pepe nero \\
        }

        \preparation{
            \begin{enumerate}
            \item Mondate la cipolla e tagliatela grossolanamente. Pulite e tagliate a pezzettini lo spicchio d'aglio. Mettete la cipolla, l'aglio e il peperoncino in una padella con l'olio e soffriggete per 4 minuti.
            \item Tagliate la zucchina a pezzettini e aggiungetela alla padella. Cucinate a fuoco medio senza coperchio mescolando spesso fino ad abbrustolirla leggermente.
            \item Tagliate in quattro i pomodorini e aggiungeteli al sugo, cucinate fino a farli appassire.
            \item Nel frattempo mettete a bollire l'acqua e cuocete la pasta.
            \item Aggiungete al sugo sale, pepe e curcuma e cuocete ancora qualche minuto.
            \item Scolate i fusilloni leggermente al dente e saltateli in padella con il sugo.
            \end{enumerate}
        }
    \end{recipe}

    \newpage
    \begin{recipe}
        [%
            preparationtime = {\unit[30]{min}},
            portion = {\unit[1]{Porzione}},
            source = {Lyon 16/01/2022},
        ]
        {Carbonara di zucca}

        \graph{
            big = CarbonaraZucca.jpeg,
        }

        \ingredients{
            \unit[120]{g} & Spaghetti \\
            1/2 & Cipolla \\
            \unit[150]{g} & Zucca \\
            \unit[2]{cucchiai} & Olio EVO \\
            \unit[1]{pizzico} & Zucchero \\
            2 & Uova \\
            \unit[50]{g} & Pecorino \\
            \unit[20]{g} & Parmigiano \\
            & Pepe nero \\
        }

        \preparation{
            \begin{enumerate}
            \item Mondate la cipolla e tagliatela grossolanamente. In un tegame scaldate l'olio e soffriggete la cipolla.
            \item Tagliate la zucca a cubetti di circa mezzo centimetro di lato e aggiungete alla padella. Fateli rosolare a fuoco medio senza coperchio aggiungendo un filo d'acqua se cominciano a bruciarsi.
            \item Nel frattempo cuocete gli spaghetti.
            \item Mentre pasta e sugo cuociono, in una ciotola mettete un uovo intero e un tuorlo, grattugiateci sopra il pecorino, il parmigiano e il pepe nero. Mescolate bene e mettete da parte.
            \item Prelevate una tazzina di acqua di cottura e poi scolate gli spaghetti al dente.
            \item Trasferite gli spaghetti nella padella e saltate con il sugo per circa 2 minuti, aggiungendo un poco di acqua di cottura per rendere il sugo più cremoso.
            \item Spegnete il fuoco e aggiungete le uova con il pepe e il formaggio. Amalgamate il tutto e impiattate. Spolverizzate con del pecorino grattugiato o a scaglie.
            \end{enumerate}
        }
    \end{recipe}

    \newpage
    \begin{recipe}
        [%
            preparationtime = {\unit[20]{min}},
            portion = {\unit[1]{Porzione}},
            source = {Lyon 11/03/2022},
        ]
        {Linguine porro e aringa}

        \graph{
            big = LinguinePorroAringa.JPG,
        }

        \ingredients{
            \unit[120]{g} & Linguine \\
            \unit[100]{g} & Porro \\
            \unit[3/2]{filetti} & Aringa affumicata \\
            \unit[1]{spicchio} & Aglio \\
            \unit[2]{cucchiai} & Olio EVO \\
            5 & Pomodorini ciliegini \\
            1/2 & Succo limone \\
            q.b. & Aneto \\
            \unit[1]{pizzico} & Peperoncino \\
        }

        \preparation{
            \begin{enumerate}
            \item Per prima cosa mettete a bollire l'acqua.
            \item Lavate e tagliate il porro, in egual quantità la parte bianca del gambo e quella verde delle foglie. Spezzettate le rondelle del gambo. Pulite uno spicchio d'aglio, tagliatelo in due e schiacciatelo con il piatto del coltello.
            \item In una padella grande fate soffriggere il porro e l'aglio con un cucchiaio d'olio.
            \item Nel frattempo tagliate a pezzettoni uno dei due filetti di aringa e a pezzettini piccoli l'altro mezzo filetto. Aggiungete alla padella.
            \item Quando l'acqua bolle salate e gettate la pasta. Nel frattempo abbassate il fuoco della padella e mescolate. Se il sugo appare secco aggiungete una tazzina di acqua di cottura.
            \item Aggiungete alla padella il succo di mezzo limone e spolverizzate di aneto. Aggiungete anche un pizzico di peperoncino.
            \item A due minuti da quando intendete scolare la pasta tagliate a metà i pomodorini e aggiungeteli al sugo.
            \item Scolate la pasta leggermente al dente e saltatela in padella per due minuti con il sugo, una tazzina di acqua di cottura e un cucchiaio d'olio. Impiattate e servite.
            \end{enumerate}
        }
    \end{recipe}

    \newpage
    \begin{recipe}
        [%
            preparationtime = {\unit[20]{min}},
            portion = {\unit[3]{Porzion1}},
            source = {Paris 30/04/2022},
        ]
        {Spaghetti alla chitarra alle pere}

        \graph{
            big = SpaghettiPere.jpeg,
        }

        \ingredients{
            \unit[400]{g} & Spaghetti alla chitarra \\
            \unit[1]{spicchio} & Aglio \\
            1 & Cipolla \\
            1 & Pera Abate \\
            \unit[2]{cucchiai} & Olio EVO \\
            1 & Peperoncino \\
            \unit[3]{rametti} & Timo \\
            \unit[1]{pizzico} & Pepe \\
            \unit[30]{g} & Formaggio fresco di capra \\
        }

        \preparation{
            \begin{enumerate}
            \item Per prima cosa mettete a bollire l'acqua.
            \item Mondate la cipolla e l'aglio e metteteli a soffriggere con l'olio e il peperoncino intero.
            \item Tagliate la pera, un quarto a pezzettoni e la parte restante a pezzi piccoli. Quando la cipolla si è dorata, aggiungete alla padella.
            \item Saltate per due minuti, poi aggiungete i rametti di timo e una tazzina da caffè di acqua di cottura. Cuocete con il coperchio.
            \item A due minuti da quando intendete scolare la pasta togliete il coperchio dal sugo e fate asciugare un po'.
            \item Pescate la pasta e mettetela nella padella del sugo. Saltate aggiungendo il formaggio a pezzettini e una spruzzata di pepe. Impiattate e servite.
            \end{enumerate}
        }
    \end{recipe}


\end{document}