% LTeX: language=it_IT

\documentclass[main.tex]{subfiles}
\graphicspath{{../Figures/Contorni/}}

\begin{document}
    \chapter{Contorni}
    
    \newpage
    \begin{recipe}
        [%
            preparationtime = {\unit[45]{min}},
            portion = {\unit[1]{Porzione}},
            source = {Lyon 6/11/2021},
        ]
        {Zucca, cipolla e mela caramellate}

        \graph{
            big = ZuccaCipollaMela.JPG
        }

        \ingredients{
            1/2 & Cipolla rossa \\
            \unit[150]{g} & Zucca \\
            1 & Mela \\
            \unit[1]{cucchiaio} & Olio EVO \\
            \unit[1/2]{bicchiere} & Acqua \\
            \unit[1]{pizzico} & Sale \\
            \unit[1]{pizzico} & Pepe nero \\
            \unit[1]{pizzico} & Zucchero \\
        }

        \preparation{
            \begin{enumerate}
            \item Mondate la cipolla e tagliatela a pezzettoni. Scaldate l'olio in una padella antiaderente e soffriggete.
            \item Tagliate a pezzettoni la zucca. Quando la cipolla comincia a diventare trasparente aggiungete la zucca alla padella. Fate saltare per un po', poi aggiungete un poco d'acqua e coprite. Cuocete a fuoco medio per 15 minuti.
            \item Tagliate la mela a pezzettoni e aggiungetela alla padella. Continuate la cottura, eventualmente aggiungendo acqua se le verdure rischiano di bruciarsi. Aggiungete il pizzico di sale.
            \item Aggiungete sale, pepe, zucchero e ancora un poco d'acqua. Mescolate e continuate la cottura finché l'acqua viene assorbita.
            \item Fate caramellare il composto e servite con una guarnizione di aceto balsamico e volendo uno o due kiwi tagliati a rondelle.
            \end{enumerate}
        }
    \end{recipe}

    \newpage
    \begin{recipe}
        [%
            preparationtime = {\unit[45]{min}},
            portion = {\unit[1]{Porzione}},
            source = {Lyon 10/11/2021},
        ]
        {Caponata autunnale}

        \graph{
            big = CaponataAutunnale.JPG
        }

        \ingredients{
            \unit[2]{spicchi} & Aglio \\
            \unit[100]{g} & Zucca \\
            1 & Patata \\
            1 & Pomodoro \\
            1/2 & Zucchina \\
            \unit[20]{g} & Olive verdi in salamoia \\
            \unit[10]{g} & Capperi in salamoia \\
            \unit[2]{cucchiai} & Olio EVO \\
            \unit[1/2]{bicchiere} & Acqua \\
            \unit[1]{cucchiaio} & Aceto di vino bianco \\
            \unit[1]{pizzico} & Sale \\
            \unit[1]{pizzico} & Peperoncino \\
            \unit[1]{pezzo} & Scorza di limone \\
        }

        \preparation{
            \begin{enumerate}
            \item Mondate uno dei due spicchi d'aglio e tagliatelo a pezzettini. Scaldate l'olio in una padella antiaderente e soffriggete brevemente.
            \item Tagliate a pezzettoni la zucca e la patate ed aggiungetele alla padella. Fate saltare per un minuto circa, poi aggiungete un poco d'acqua e coprite. Cuocete a fuoco medio per 10 minuti.
            \item Tagliate la zucchina a pezzettoni e aggiungetela alla padella. Continuate la cottura per altri 10 minuti, eventualmente aggiungendo acqua se le verdure rischiano di bruciarsi. Aggiungete il pizzico di sale.
            \item Tagliate il pomodoro a pezzettoni e aggiungetelo alla padella insieme alle olive e ai capperi.
            \item Sminuzzate la scorza di limone e tagliate a pezzetti il secondo spicchio di aglio. Aggiungete entrambi alla padella insieme alla spolverata di peperoncino e al cucchiaio di aceto.
            \item Cuocete ancora per un poco finché le verdure cominciano a spappolarsi.
            \item Servite possibilmente accompagnando con del pane o una piadina. Se volete potete aggiungere al piatto una piccola degustazione di formaggi con miele o confetture.
            \end{enumerate}
        }
    \end{recipe}


\end{document}