% LTeX: language=it_IT

\documentclass[main.tex]{subfiles}
\graphicspath{{Figures/Secondi/}}

\begin{document}
    \chapter{Secondi piatti}

    \newpage
    \begin{recipe}
        [%
            preparationtime = {\unit[45]{min}},
            portion = {\unit[1]{Porzione}},
            source = {Lyon 16/10/2021},
        ]
        {Roesti di patate e zucca al Camembert}

        \graph{
            big = RoestiPatateZucca.JPG
        }

        \ingredients{
            1/2 & Cipolla \\
            \unit[150]{g} & Zucca \\
            1 & patata \\
            \unit[50]{g} & Camembert \\
            1 & Uovo \\
            \unit[1]{cucchiaio} & Olio EVO \\
            \unit[1]{bicchiere} & Acqua \\
            \unit[1]{pizzico} & Sale \\
            \unit[1]{pizzico} & Pepe nero \\
            \unit[1]{pizzico} & Zucchero \\
        }

        \preparation{
            \begin{enumerate}
            \item Mondate la cipolla e tagliatela a pezzettini piccoli. Scaldate l'olio in una padella antiaderente e soffriggete.
            \item Tagliate a cubetti piccoli la zucca e le patate ed aggiungetele alla padella. Fate saltare per un minuto circa, poi aggiungete gradualmente l'acqua. Il quantitativo di acqua dipende da quanto è alto il fuoco e se cucinate con o senza coperchio. L'importante è non annegare il contenuto ma neanche bruciarlo. Aggiungere sale, pepe e zucchero e mescolare.
            \item Quando le patate e la zucca si sono ammorbidite fate evaporare l'acqua in eccesso (lasciate comunque che il composto sia umido) e poi aggiungete il Camembert. Fatelo sciogliere e mescolate il tutto, mantecando.
            \item Aggiungete l'uovo e mescolate per amalgamare.
            \item Smettete di mescolare e aspettate che il composto assuma una consistenza pseudo-solida.
            \item Servite accompagnando con un insalata verde semplice (e.g. Valerianella).
            \end{enumerate}
        }
    \end{recipe}


\end{document}