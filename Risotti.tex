\documentclass[main.tex]{subfiles}

\begin{document}
    \chapter{Risotti}

    \newpage
    \begin{recipe}
        [%
            preparationtime = {\unit[1]{h}},
            portion = {\unit[1]{Porzione}},
            source = {Lyon 5/12/2021},
        ]
        {Risotto di zucca al Roquefort e noci}

        \graph{
            big = RisottoZuccaRoquefort.JPG
        }

        \ingredients{
            1/2 & Cipolla \\
            \unit[200]{g} & Zucca \\
            \unit[100]{g} & Riso Carnaroli \\
            \unit[70]{g} & Roquefort \\
            \unit[25]{g} & Burro \\
            5 & Noci \\
            \unit[2]{cucchiai} & Olio EVO \\
            \unit[2]{bicchieri} & Acqua \\
            1/2 & Dado vegetale \\
            \unit[1]{pizzico} & Sale \\
            \unit[1]{pizzico} & Zucchero \\
            & Pepe nero \\
        }

        \preparation{
            \begin{enumerate}
            \item Mondate la cipolla e tagliatela finemente. Mettetela in un tegame con l'olio e soffriggete per 2 minuti. Abbassate il fuoco al minimo e lasciate cuocere per altri 10 minuti.
            \item In un pentolino scaldate l'acqua e scioglietevi il dado vegetale.
            \item Pulite e tagliate a pezzettini piccoli la zucca, aggiungetela al tegame della cipolla. Alzate il fuoco e rosolate per qualche minuto. Dopodiché aggiungete due mestoli di brodo caldo, il sale, il pepe e lo zucchero. Mescolate, abbassate il fuoco e coprite. Lasciate cuocere per circa 20 minuti.
            \item Nel frattempo scaldate una padella larga e tostatevi brevemente il riso.
            \item Sgusciate i gherigli delle noci e fatene una grossolana granella (per esempio con un mortaio).
            \item La zucca dovrebbe essere ora morbida, e i pezzi dovrebbero spappolarsi quando mescolate. Aggiungete il riso e altri due mestoli di brodo. Fate cuocere per il tempo necessario al riso.
            \item A cottura quasi ultimata aggiungete il burro e il Roquefort a pezzettini, facendoli sciogliere e mantecando così il risotto.
            \item Aggiungete la granella di noci e impiattate. Come decorazione stanno bene due foglioline di menta.
            \end{enumerate}
        }
    \end{recipe}
\end{document}