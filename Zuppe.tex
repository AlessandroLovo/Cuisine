\documentclass[main.tex]{subfiles}
\graphicspath{{Figures/Zuppe/}}

\begin{document}
    \chapter{Zuppe}

    \newpage
    \begin{recipe}
        [%
            preparationtime = {\unit[1]{h}},
            portion = {\unit[1]{Porzione}},
            source = {Lyon 14/01/2022},
        ]
        {Crema di zucca con formaggio di capra}

        \graph{
            big = CremaZucca.jpeg
        }

        \ingredients{
            1 & Cipolla \\
            1 & Patata \\
            \unit[300]{g} & Zucca \\
            \unit[2]{cucchiai} & Olio EVO \\
            \unit[1/2]{litro} & Acqua \\
            1 & Dado vegetale \\
            \unit[1]{tazzina} & Vino bianco \\
            \unit[1]{pizzico} & Sale \\
            \unit[1]{pizzico} & Zucchero \\
            & Pepe nero \\
            & Noce moscata \\
            \unit[50]{g} & Formaggio di capra \\
        }

        \preparation{
            \begin{enumerate}
            \item Mondate la cipolla e tagliatela grossolanamente. In un tegame scaldate l'olio e soffriggete la cipolla. Una volta rosolata sfumate con il vino bianco.
            \item Nel frattempo scaldate l'acqua con il dado vegetale per preparare il brodo.
            \item Tagliate la zucca e la patata a pezzetti e aggiungete al tegame. Rosolate per qualche minuto.
            \item Aggiungete il brodo e cuocete a fuoco medio mescolando spesso per circa 30-45 minuti, finché la zucca e la patata si squagliano al mescolare.
            \item Aggiungete sale, zucchero, noce moscata e pepe. Mescolate bene e versate in un piatto fondo.
            \item Tagliate a pezzetti il formaggio di capra e adagiatelo sulla crema di zucca.
            \item Volendo aggiungete sul piatto dei pezzetti di pane o dei crostini.
            \end{enumerate}
        }
    \end{recipe}

    \newpage
    \begin{recipe}
        [%
            preparationtime = {\unit[40]{min}},
            portion = {\unit[1]{Porzione}},
            source = {Lyon 06/03/2022},
        ]
        {Vellutata di zucca con gamberi ubriachi}

        \graph{
            big = VellutataZuccaGamberi.JPG,
        }

        \ingredients{
            1 & Cipolla \\
            \unit[200]{g} & Zucca \\
            \unit[2]{cucchiai} & Olio EVO \\
            \unit[1]{pizzico} & Zucchero \\
            1 & Dado vegetale \\
            \unit[1/2]{litro} & Acqua \\
            \unit[70]{g} & Gamberetti \\
            \unit[1/2]{bicchiere} & Vino bianco \\
            q.b. & Paprika dolce \\
            q.b. & Zenzero \\
        }

        \preparation{
            \begin{enumerate}
            \item In un pentolino mettete a bollire l'acqua con il dado vegetale.
            \item Mondate la cipolla e la zucca. Tagliate entrambe a pezzettini e mettete in una pentola con un l'olio a soffriggere.
            \item Aggiungete due mestoli di brodo vegetale e lo zucchero e continuate la cottura a fuoco medio continuando ad aggiungere brodo un poco alla volta per circa 30 min.
            \item Passato questo tempo mettete i gamberetti in una padella già calda con un filo d'olio, aggiungete il vino bianco e mescolate. Cuocete finché il sugo dei gamberi non si rapprende. Lasciatelo umido in ogni caso.
            \item Frullate la zucca con per ottenere una crema, mescolateci lo zenzero e la paprika e trasferite in un piatto. Aggiungete i gamberi.
            \item Potete aggiungere anche dei crostini, un filo d'olio o di aceto balsamico e qualche foglia di prezzemolo.
            \end{enumerate}
        }
    \end{recipe}


\end{document}