\documentclass[11pt, a4 paper]{article}

\usepackage[top=1cm,bottom=2cm,left=2cm,right=2cm]{geometry}
\usepackage[italian]{babel}
\usepackage[T1]{fontenc}
\usepackage[utf8]{inputenc}
\usepackage{url}
\usepackage{hyperref}
\usepackage{xcookybooky}

\graphicspath{{Figures/}}

\begin{document}

\tableofcontents

\newpage
\begin{recipe}
  [%
    preparationtime = {\unit[45]{min}},
    portion = {\unit[1]{Porzione}},
    source = {Lyon 16/10/2021},
  ]
  {Roesti di patate e zucca al Camembert}

  \graph{
    big = RoestiPatateZucca.JPG
  }

  \ingredients{
    1/2 & Cipolla \\
    \unit[150]{g} & Zucca \\
    1 & patata \\
    \unit[50]{g} & Camembert \\
    1 & Uovo \\
    \unit[1]{cucchiaio} & Olio EVO \\
    \unit[1]{bicchiere} & Acqua \\
    \unit[1]{pizzico} & Sale \\
    \unit[1]{pizzico} & Pepe nero \\
    \unit[1]{pizzico} & Zucchero \\
  }

  \preparation{
    \begin{enumerate}
      \item Mondate la cipolla e tagliatela a pezzettini piccoli. Scaldate l'olio in una padella antiaderente e soffriggete.
      \item Tagliate a cubetti piccoli la zucca e le patate ed aggiungetele alla padella. Fate saltare per un minuto circa, poi aggiungete gradualmente l'acqua. Il quantitativo di acqua dipende da quanto è alto il fuoco e se cucinate con o senza coperchio. L'importante è non annegare il contenuto ma neanche bruciarlo. Aggiungere sale, pepe e zucchero e mescolare.
      \item Quando le patate e la zucca si sono ammorbidite fate evaporare l'acqua in eccesso (lasciate comunque che il composto sia umido) e poi aggiungete il Camembert. Fatelo sciogliere e mescolate il tutto, mantecando.
      \item Aggiungete l'uovo e mescolate per amalgamare.
      \item Smettete di mescolare e aspettate che il composto assuma una consistenza pseudo-solida.
      \item Servite accompagnando con un insalata verde semplice (e.g. Valerianella).
    \end{enumerate}
  }
\end{recipe}

\newpage
\begin{recipe}
  [%
    preparationtime = {\unit[45]{min}},
    portion = {\unit[1]{Porzione}},
    source = {Lyon 20/10/2021},
  ]
  {Spaghetti alla zucca e zucchine}

  \graph{
    big = SpaghettiZuccaZucchine.JPG
  }

  \ingredients{
    \unit[120]{g} & Spaghetti \\
    1/2 & Cipolla \\
    \unit[150]{g} & Zucca \\
    1 & Zucchina \\
    \unit[1]{cucchiaio} & Olio EVO \\
    \unit[1]{bicchiere} & Acqua \\
    \unit[1]{pizzico} & Sale \\
    \unit[1]{pizzico} & Zucchero \\
    \unit[1]{spolverata} & Peperoncino \\
    \unit[1]{pezzo} & Scorza di limone \\
  }

  \preparation{
    \begin{enumerate}
      \item Mondate la cipolla e tagliatela a pezzettini piccoli. Scaldate l'olio in una padella antiaderente e soffriggete.
      \item Tagliate a cubetti piccoli la zucca e aggiungetela alla padella. Fate saltare per un minuto circa, poi aggiungete gradualmente l'acqua. Il quantitativo di acqua dipende da quanto è alto il fuoco e se cucinate con o senza coperchio. L'importante è non annegare il contenuto ma neanche bruciarlo. Nel frattempo tagliate la zucchina a julienne e scaldate l'acqua per la pasta.
      \item Quando la zucca comincia ad ammorbidirsi aggiungete la zucchina, sale, zucchero e peperoncino.
      \item Finché vi dedicate alla cottura degli spaghetti controllate che le zucchine non si spappolino. Se cominciano a diventare troppo molli, semplicemente spegnete il fuoco.
      \item Finché la pasta cuoce sminuzzate la scorza di limone e aggiungetela al sugo.
      \item Quando gli spaghetti sono pronti scolateli (meglio un po' al dente) e poi saltateli velocemente con il sugo preparato.
    \end{enumerate}
  }

\end{recipe}

\newpage
\begin{recipe}
  [%
    preparationtime = {\unit[30]{min}},
    portion = {\unit[1]{Porzione}},
    source = {Lyon 22/10/2021},
  ]
  {Spaghetti cacio e kiwi}

  \graph{
    big = SpaghettiCacioKiwi.JPG
  }

  \ingredients{
    \unit[120]{g} & Spaghetti \\
    1/4 & Cipolla \\
    2 & Kiwi \\
    \unit[40]{g} & Pecorino sardo stagionato \\
    \unit[1]{cucchiaio} & Olio EVO \\
    \unit[1]{pizzico} & Zucchero \\
    \unit[1]{spolverata} & Peperoncino \\
    \unit[1]{rametto} & Timo \\
  }

  \preparation{
    \begin{enumerate}
      \item Mondate la cipolla e tagliatela a pezzettini. Scaldate l'olio in un pentolino antiaderente e soffriggete.
      \item Sbucciate e tagliate a pezzettini i due kiwi aggiungeteli alla padella. Fate saltare per un minuto circa, poi aggiungete un goccio d'acqua, lo zucchero il timo e il peperoncino. Mescolate e coprite con un coperchio. Cuocete a fuoco medio basso finché non ottenete una pseudo-poltiglia.
      \item Nel frattempo cuocete la pasta e grattugiate il pecorino.
      \item Quando la pasta è pronta scolatela e impiattate mescolandovi il sugo e il pecorino grattuggiato.
    \end{enumerate}
  }

\end{recipe}

\newpage
\begin{recipe}
  [%
    preparationtime = {\unit[45]{min}},
    portion = {\unit[1]{Porzione}},
    source = {Lyon 6/11/2021},
  ]
  {Zucca, cipolla e mela caramellate}

  \graph{
    big = ZuccaCipollaMela.JPG
  }

  \ingredients{
    1/2 & Cipolla rossa \\
    \unit[150]{g} & Zucca \\
    1 & Mela \\
    \unit[1]{cucchiaio} & Olio EVO \\
    \unit[1/2]{bicchiere} & Acqua \\
    \unit[1]{pizzico} & Sale \\
    \unit[1]{pizzico} & Pepe nero \\
    \unit[1]{pizzico} & Zucchero \\
  }

  \preparation{
    \begin{enumerate}
      \item Mondate la cipolla e tagliatela a pezzettoni. Scaldate l'olio in una padella antiaderente e soffriggete.
      \item Tagliate a pezzettoni la zucca. Quando la cipolla comincia a diventare trasparente aggiungete la zucca alla padella. Fate saltare per un po', poi aggiungete un poco d'acqua e coprite. Cuocete a fuoco medio per 15 minuti.
      \item Tagliate la mela a pezzettoni e aggiungetela alla padella. Continuate la cottura, eventualmente aggiungendo acqua se le verdure rischiano di bruciarsi. Aggiungete il pizzico di sale.
      \item Aggiungete sale, pepe, zucchero e ancora un poco d'acqua. Mescolate e continuate la cottura finché l'acqua viene assorbita.
      \item Fate caramellare il composto e servite con una guarnizione di aceto balsamico e volendo uno o due kiwi tagliati a rondelle.
    \end{enumerate}
  }
\end{recipe}

\newpage
\begin{recipe}
  [%
    preparationtime = {\unit[45]{min}},
    portion = {\unit[1]{Porzione}},
    source = {Lyon 10/11/2021},
  ]
  {Caponata autunnale}

  \graph{
    big = CaponataAutunnale.JPG
  }

  \ingredients{
    \unit[2]{spicchi} & Aglio \\
    \unit[100]{g} & Zucca \\
    1 & Patata \\
    1 & Pomodoro \\
    1/2 & Zucchina \\
    \unit[20]{g} & Olive verdi in salamoia \\
    \unit[10]{g} & Capperi in salamoia \\
    \unit[2]{cucchiai} & Olio EVO \\
    \unit[1/2]{bicchiere} & Acqua \\
    \unit[1]{cucchiaio} & Aceto di vino bianco \\
    \unit[1]{pizzico} & Sale \\
    \unit[1]{pizzico} & Peperoncino \\
    \unit[1]{pezzo} & Scorza di limone \\
  }

  \preparation{
    \begin{enumerate}
      \item Mondate uno dei due spicchi d'aglio e tagliatelo a pezzettini. Scaldate l'olio in una padella antiaderente e soffriggete brevemente.
      \item Tagliate a pezzettoni la zucca e la patate ed aggiungetele alla padella. Fate saltare per un minuto circa, poi aggiungete un poco d'acqua e coprite. Cuocete a fuoco medio per 10 minuti.
      \item Tagliate la zucchina a pezzettoni e aggiungetela alla padella. Continuate la cottura per altri 10 minuti, eventualmente aggiungendo acqua se le verdure rischiano di bruciarsi. Aggiungete il pizzico di sale.
      \item Tagliate il pomodoro a pezzettoni e aggiungetelo alla padella insieme alle olive e ai capperi.
      \item Sminuzzate la scorza di limone e tagliate a pezzetti il secondo spicchio di aglio. Aggiungete entrambi alla padella insieme alla spolverata di peperoncino e al cucchiaio di aceto.
      \item Cuocete ancora per un poco finché le verdure cominciano a spappolarsi.
      \item Servite possibilmente accompagnando con del pane o una piadina. Se volete potete aggiungere al piatto una piccola degustazione di formaggi con miele o confetture.
    \end{enumerate}
  }
\end{recipe}

\newpage
\begin{recipe}
  [%
    preparationtime = {\unit[10]{min}},
    portion = {\unit[1]{Porzione}},
    source = {Lyon 29/11/2021},
  ]
  {Insalata Danese}

  \graph{
    big = InsalataDanese.JPG
  }

  \ingredients{
    1/8 & Cipolla \\
    \unit[100]{g} & Aringa affumicata \\
    \unit[150]{g} & Valerianella \\
    \unit[10]{foglie} & Basilico \\
    10-15 & Olive verdi \\
    \unit[2]{cucchiai} & Capperi in salamoia \\
    \unit[1.5]{cucchiai} & Olio EVO \\
    \unit[1]{cucchiaino} & Miele \\
    \unit[1]{spruzzata} & Aneto \\
  }

  \preparation{
    \begin{enumerate}
      \item Mondate la cipolla e tagliatela a striscioline.
      \item Lavate e centrifugate la valerianella e il basilico.
      \item Tagliate a pezzettoni l'aringa affumicata.
      \item In una ciotola o piatto fondo mescolate la valerianella, il basilico, la cipolla, le olive, i capperi e l'aringa. In una tazzina mescolate il miele con l'olio e l'aneto, poi condite l'insalata.
    \end{enumerate}
  }
\end{recipe}

\newpage
\begin{recipe}
  [%
    preparationtime = {\unit[1]{h}},
    portion = {\unit[1]{Porzione}},
    source = {Lyon 5/12/2021},
  ]
  {Risotto di zucca al Roquefort e noci}

  \graph{
    big = RisottoZuccaRoquefort.JPG
  }

  \ingredients{
    1/2 & Cipolla \\
    \unit[200]{g} & Zucca \\
    \unit[100]{g} & Riso Carnaroli \\
    \unit[70]{g} & Roquefort \\
    \unit[25]{g} & Burro \\
    5 & Noci \\
    \unit[2]{cucchiai} & Olio EVO \\
    \unit[2]{bicchieri} & Acqua \\
    1/2 & Dado vegetale \\
    \unit[1]{pizzico} & Sale \\
    \unit[1]{pizzico} & Zucchero \\
    & Pepe nero \\
  }

  \preparation{
    \begin{enumerate}
      \item Mondate la cipolla e tagliatela finemente. Mettetela in un tegame con l'olio e soffriggete per 2 minuti. Abbassate il fuoco al minimo e lasciate cuocere per altri 10 minuti.
      \item In un pentolino scaldate l'acqua e scioglietevi il dado vegetale.
      \item Pulite e tagliate a pezzettini piccoli la zucca, aggiungetela al tegame della cipolla. Alzate il fuoco e rosolate per qualche minuto. Dopodiché aggiungete due mestoli di brodo caldo, il sale, il pepe e lo zucchero. Mescolate, abbassate il fuoco e coprite. Lasciate cuocere per circa 20 minuti.
      \item Nel frattempo scaldate una padella larga e tostatevi brevemente il riso.
      \item Sgusciate i gherigli delle noci e fatene una grossolana granella (per esempio con un mortaio).
      \item La zucca dovrebbe essere ora morbida, e i pezzi dovrebbero spappolarsi quando mescolate. Aggiungete il riso e altri due mestoli di brodo. Fate cuocere per il tempo necessario al riso.
      \item A cottura quasi ultimata aggiungete il burro e il Roquefort a pezzettini, facendoli sciogliere e mantecando così il risotto.
      \item Aggiungete la granella di noci e impiattate. Come decorazione stanno bene due foglioline di menta.
    \end{enumerate}
  }
\end{recipe}

\newpage
\begin{recipe}
  [%
    preparationtime = {\unit[1]{h}},
    portion = {\unit[1]{Porzione}},
    source = {Lyon 13/01/2022},
  ]
  {Fusilloni con zucchine alla curcuma}

  \graph{
    big = FusilloniZucchineCurcuma.jpeg
  }

  \ingredients{
    \unit[120]{g} & Fusilloni \\
    1/2 & Cipolla \\
    1/2 & Zucchina \\
    \unit[1]{spicchio} Aglio \\
    10 & Pomodorini \\
    \unit[2]{cucchiai} & Olio EVO \\
    \unit[1]{pizzico} & Sale \\
    \unit[1]{pizzico} & Peperoncino \\
    \unit[1]{cucchiaino} & Curcuma \\
    & Pepe nero \\
  }

  \preparation{
    \begin{enumerate}
      \item Mondate la cipolla e tagliatela grossolanamente. Pulite e tagliate a pezzettini lo spicchio d'aglio. Mettete la cipolla, l'aglio e il peperoncino in una padella con l'olio e soffriggete per 4 minuti.
      \item Tagliate la zucchina a pezzettini e aggiungetela alla padella. Cucinate a fuoco medio senza coperchio mescolando spesso fino ad abbrustolirla leggermente.
      \item Tagliate in quattro i pomodorini e aggiungeteli al sugo, cucinate fino a farli appassire.
      \item Nel frattempo mettete a bollire l'acqua e cuocete la pasta.
      \item Aggiungete al sugo sale, pepe e curcuma e cuocete ancora qualche minuto.
      \item Scolate i fusilloni leggermente al dente e saltateli in padella con il sugo.
    \end{enumerate}
  }
\end{recipe}

\newpage
\begin{recipe}
  [%
    preparationtime = {\unit[1]{h}},
    portion = {\unit[1]{Porzione}},
    source = {Lyon 14/01/2022},
  ]
  {Crema di zucca con formaggio di capra}

  \graph{
    big = CremaZucca.jpeg
  }

  \ingredients{
    1 & Cipolla \\
    1 & Patata \\
    \unit[300]{g} & Zucca \\
    \unit[2]{cucchiai} & Olio EVO \\
    \unit[1/2]{litro} & Acqua \\
    1 & Dado vegetale \\
    \unit[1]{tazzina} & Vino bianco \\
    \unit[1]{pizzico} & Sale \\
    \unit[1]{pizzico} & Zucchero \\
    & Pepe nero \\
    & Noce moscata \\
    \unit[50]{g} & Formaggio di capra \\
  }

  \preparation{
    \begin{enumerate}
      \item Mondate la cipolla e tagliatela grossolanamente. In un tegame scaldate l'olio e soffriggete la cipolla. Una volta rosolata sfumate con il vino bianco.
      \item Nel frattempo scaldate l'acqua con il dado vegetale per preparare il brodo.
      \item Tagliate la zucca e la patata a pezzetti e aggiungete al tegame. Rosolate per qualche minuto.
      \item Aggiungete il brodo e cuocete a fuoco medio mescolando spesso per circa 30-45 minuti, finché la zucca e la patata si squagliano al mescolare.
      \item Aggiungete sale, zucchero, noce moscata e pepe. Mescolate bene e versate in un piatto fondo.
      \item Tagliate a pezzetti il formaggio di capra e adagiatelo sulla crema di zucca.
      \item Volendo aggiungete sul piatto dei pezzetti di pane o dei crostini.
    \end{enumerate}
  }
\end{recipe}

\newpage
\begin{recipe}
  [%
    preparationtime = {\unit[30]{min}},
    portion = {\unit[1]{Porzione}},
    source = {Lyon 16/01/2022},
  ]
  {Carbonara di zucca}

  \graph{
    big = CarbonaraZucca.jpeg,
  }

  \ingredients{
    \unit[120]{g} & Spaghetti \\
    1/2 & Cipolla \\
    \unit[150]{g} & Zucca \\
    \unit[2]{cucchiai} & Olio EVO \\
    \unit[1]{pizzico} & Zucchero \\
    2 & Uova \\
    \unit[50]{g} & Pecorino \\
    \unit[20]{g} & Parmigiano \\
    & Pepe nero \\
  }

  \preparation{
    \begin{enumerate}
      \item Mondate la cipolla e tagliatela grossolanamente. In un tegame scaldate l'olio e soffriggete la cipolla.
      \item Tagliate la zucca a cubetti di circa mezzo centimetro di lato e aggiungete alla padella. Fateli rosolare a fuoco medio senza coperchio aggiungendo un filo d'acqua se cominciano a bruciarsi.
      \item Nel frattempo cuocete gli spaghetti.
      \item Mentre pasta e sugo cuociono, in una ciotola mettete un uovo intero e un tuorlo, grattugiateci sopra il pecorino, il parmigiano e il pepe nero. Mescolate bene e mettete da parte.
      \item Prelevate una tazzina di acqua di cottura e poi scolate gli spaghetti al dente.
      \item Trasferite gli spaghetti nella padella e saltate con il sugo per circa 2 minuti, aggiungendo un poco di acqua di cottura per rendere il sugo più cremoso.
      \item Spegnete il fuoco e aggiungete le uova con il pepe e il formaggio. Amalgamate il tutto e impiattate. Spolverizzate con del pecorino grattugiato o a scaglie.
    \end{enumerate}
  }
\end{recipe}

\newpage
\begin{recipe}
  [%
    preparationtime = {\unit[40]{min}},
    portion = {\unit[1]{Porzione}},
    source = {Lyon 06/03/2022},
  ]
  {Vellutata di zucca con gamberi ubriachi}

  \graph{
    big = VellutataZuccaGamberi.JPG,
  }

  \ingredients{
    1 & Cipolla \\
    \unit[200]{g} & Zucca \\
    \unit[2]{cucchiai} & Olio EVO \\
    \unit[1]{pizzico} & Zucchero \\
    1 & Dado vegetale \\
    \unit[1/2]{litro} & Acqua \\
    \unit[70]{g} & Gamberetti \\
    \unit[1/2]{bicchiere} & Vino bianco \\
    q.b. & Paprika dolce \\
    q.b. & Zenzero \\
  }

  \preparation{
    \begin{enumerate}
      \item In un pentolino mettete a bollire l'acqua con il dado vegetale.
      \item Mondate la cipolla e la zucca. Tagliate entrambe a pezzettini e mettete in una pentola con un l'olio a soffriggere.
      \item Aggiungete due mestoli di brodo vegetale e lo zucchero e continuate la cottura a fuoco medio continuando ad aggiungere brodo un poco alla volta per circa 30 min.
      \item Passato questo tempo mettete i gamberetti in una padella già calda con un filo d'olio, aggiungete il vino bianco e mescolate. Cuocete finché il sugo dei gamberi non si rapprende. Lasciatelo umido in ogni caso.
      \item Frullate la zucca con per ottenere una crema, mescolateci lo zenzero e la paprika e trasferite in un piatto. Aggiungete i gamberi.
      \item Potete aggiungere anche dei crostini, un filo d'olio o di aceto balsamico e qualche foglia di prezzemolo.
    \end{enumerate}
  }
\end{recipe}

\newpage
\begin{recipe}
  [%
    preparationtime = {\unit[20]{min}},
    portion = {\unit[1]{Porzione}},
    source = {Lyon 11/03/2022},
  ]
  {Linguine porro e aringa}

  % \graph{
  %   big = LinguinePorroAringa.jpeg,
  % }

  \ingredients{
    \unit[120]{g} & Linguine \\
    \unit[100]{g} & Porro \\
    \unit[3/2]{filetti} & Aringa affumicata \\
    \unit[1]{spicchio} & Aglio \\
    \unit[2]{cucchiai} & Olio EVO \\
    5 & Pomodorini ciliegini \\
    1/2 & Succo limone \\
    q.b. & Aneto \\
    \unit[1]{pizzico} & Peperoncino \\
  }

  \preparation{
    \begin{enumerate}
      \item Per prima cosa mettete a bollire l'acqua.
      \item Lavate e tagliate il porro, in egual quantità la parte bianca del gambo e quella verde delle foglie. Spezzettate le rondelle del gambo. Pulite uno spicchio d'aglio, tagliatelo in due e schiacciatelo con il piatto del coltello.
      \item In una padella grande fate soffriggere il porro e l'aglio con un cucchiaio d'olio.
      \item Nel frattempo tagliate a pezzettoni uno dei due filetti di aringa e a pezzettini piccoli l'altro mezzo filetto. Aggiungete alla padella.
      \item Quando l'acqua bolle salate e gettate la pasta. Nel frattempo abbassate il fuoco della padella e mescolate. Se il sugo appare secco aggiungete una tazzina di acqua di cottura.
      \item Aggiungete alla padella il succo di mezzo limone e spolverizzate di aneto. Aggiungete anche un pizzico di peperoncino.
      \item A due minuti da quando intendete scolare la pasta tagliate a metà i pomodorini e aggiungeteli al sugo.
      \item Scolate la pasta leggermente al dente e saltatela in padella per due minuti con il sugo, una tazzina di acqua di cottura e un cucchiaio d'olio. Impiattate e servite.
    \end{enumerate}
  }
\end{recipe}

\end{document}
